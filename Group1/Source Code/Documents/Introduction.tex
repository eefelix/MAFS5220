\section{Introduction}

In OTC market, abundant financial products are traded frequently and the price is always concerned by each trader. In traditional, risk neutral approach is applied to get arbitrage price of financial contracts. One of the problems is that no-arbitrage price is just a default free price which assumes all entities will fulfill its obligation. However, counterparty default risk widely exists in OTC market so taking counterparty risk into consideration is necessary when one tries to price an OTC product �C this leads to the computation of CVA family, including CVA, DVA, FVA, etc.

Besides from the pricing point of view to analysis counterparty risk, financial institution want to integrate all results into a simple number to indicate how much reserves they should put aside which this leads to computations of VaR and Credit VaR. These requirements motivates us to build an application to compute the counterparty risk factors like Credit VaR, CVA, DVA and FVA, with Wrong Way Risk (WWR) and collateral into consideration.

Risk Analysis Tool is a platform that we first attempt to make it into use. We create the computation supporting Interest Rate Swap, Equity Swap and Cross Currency Swap products, and a portfolio with them. The computing method is determined case by case, either using analytical solution, or Monte Carlo method. Given the fact that intensive computation such as Monte Carlo is frequently needed, we create the platform as a cloud service that would potentially be accessed via various ways, including web, mobile application, and desktop clients. Our initial version includes a mobile application for Android platform. Support for other client-side access would be done in future. Our pricing components utilize the QuantLib and our own implementation tries to be consistent with the QuantLib design. We plan to make our extensions be a part of the next version of QuantLib.

The rest of the document is organized as follow: Section~\ref{sec:background} introduces necessary background knowledge for counterparty credit risk and funding valuation adjustment, Section~\ref{sec:scope} defines the scope of the project including the products and features to be covered, Section~\ref{sec:design} represents the design of the compute engine, Section~\ref{sec:implementation} discusses important implementation details, traps and pitfalls in the system, Section~\ref{sec:background} discusses about the testing methodology that ensures the quality of our project.

For more details of our implementation including the API and class reference and testing status, additional documents are provided as appendixes.
